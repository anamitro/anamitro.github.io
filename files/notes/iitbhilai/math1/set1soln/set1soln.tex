\documentclass[a4paper,11pt]{article}
\usepackage[T1]{fontenc}
\usepackage[utf8]{inputenc}
\usepackage{lmodern}
\usepackage{xcolor}
\usepackage{textcomp}
%\usepackage{newspaper}
%\date{\today}
%\currentvolume{1430}
%\currentissue{6}
%\usepackage{times}
\usepackage{graphicx}
\usepackage{multicol}
\usepackage{etoolbox}
\usepackage{picinpar}
%uasage of picinpar:
%\begin{window}[1,l,\includegraphics{},caption]xxxxx\end{window}
\usepackage{times}
\usepackage[T1]{fontenc}
\usepackage[utf8]{inputenc}
%\usepackage[nohead, nomarginpar, margin=1in, foot=.25in]{geometry}
\usepackage{lmodern}
%\usepackage{fontspec}
\usepackage{setspace}
%\usepackage{fancyhdr}
\usepackage{everyshi}
\usepackage{pdfpages}
\usepackage{indentfirst}
\usepackage[hyphens]{url}
\usepackage{hyperref}
\usepackage{pdfpages}
\usepackage{pgfpages}
\usepackage{amssymb}
\usepackage{amsmath}
\usepackage[notextcomp]{stix}
 \usepackage{paracol}
 \usepackage{stmaryrd}
 \usepackage{xurl}
 \usepackage{wasysym}
\usepackage[left=2cm, right=2cm, top=2cm, bottom=2cm]{geometry}
%\pagestyle{fancy}
%\lhead{Curriculum vitae, Anamitro Biswas}
%\chead{}
%\rhead{}
%\lfoot{}
%\cfoot{\thepage}
%\rfoot{}
%\makeatletter
%\renewcommand{\@chapapp}{Document}
%\makeatother
%\renewcommand\refname{\enhead\color{red}{References:}}
%\renewcommand{\headrulewidth}{0.4pt}
%\renewcommand{\footrulewidth}{0.4pt}
%\newfontfamily\bn{TiroBangla-Regular.ttf}[Script=Bengali]
%\newfontfamily\enhead{RubikDoodleShadow-Regular.ttf}[Script=Latin]
%\setmainfont{cmunrm.ttf}[Script=Latin]
%\newfontfamily\bn{TiroBangla-Regular.ttf}[Script=Bengali]
%\newfontfamily\latinfont{OfenbacherSchwabCAT.ttf}[Script=Latin]


%BRACKETS:
\newcommand{\lb}{\left(}
\newcommand{\rb}{\right)}
\newcommand{\lc}{\left\lbrace}
\newcommand{\rc}{\right\rbrace}
\newcommand{\ls}{\left[}
\newcommand{\rs}{\right]}
\renewcommand{\la}{\left|}
\renewcommand{\ra}{\right|}

%GREEK UPPERCASE:
\newcommand{\Lm}{\Lambda}
\newcommand{\G}{\Gamma}
\newcommand{\D}{\Delta}
\newcommand{\Th}{\Theta}

%GREEK LOWERCASE:
\newcommand{\m}{\mu}
\newcommand{\lm}{\lambda}
\newcommand{\vs}{\varsigma}
\newcommand{\g}{\gamma}
\renewcommand{\t}{\tau}
\renewcommand{\a}{\alpha}
\renewcommand{\b}{\beta}
\newcommand{\e}{\varepsilon}
\newcommand{\p}{\pi}
\newcommand{\dt}{\delta}
\newcommand{\f}{\varphi}
\newcommand{\z}{\zeta}
\newcommand{\s}{\sigma}

%SET THEORY:
\newcommand{\cl}{\overline}
\newcommand{\intr}{^\circ}
\newcommand{\ir}{^\text{int}}
\newcommand{\comp}{^C}

%MATHCAL:
\newcommand{\A}{\mathcal{A}}
\newcommand{\B}{\mathcal{B}}
\newcommand{\C}{\mathcal{C}}
\newcommand{\cD}{\mathcal{D}}
\newcommand{\E}{\mathcal{E}}
\newcommand{\F}{\mathcal{F}}
\newcommand{\cG}{\mathcal{G}}
\newcommand{\Hy}{\mathcal{H}}
\newcommand{\I}{\mathcal{I}}
\newcommand{\J}{\mathcal{J}}
\newcommand{\List}{\mathcal{L}}
\newcommand{\M}{\mathcal{M}}
\newcommand{\N}{\mathcal{N}}
\newcommand{\R}{\mathcal{R}}
\newcommand{\Pa}{\mathcal{P}}
\newcommand{\La}{\mathcal{L}}
\newcommand{\Ta}{\mathcal{T}}
\newcommand{\mcS}{\mathcal{S}}
\newcommand{\V}{\mathcal{V}}
\newcommand{\X}{\mathcal{X}}
\newcommand{\Y}{\mathcal{Y}}

%MATHFRAK:
\newcommand{\Mf}{\mathfrak{M}}
\newcommand{\Nf}{\mathfrak{N}}
\newcommand{\gf}{\mathfrak{g}}
\newcommand{\mfi}{\mathfrak{i}}
\newcommand{\mfj}{\mathfrak{j}}
\newcommand{\mfa}{\mathfrak{A}}

%FIELDS:
\newcommand{\reals}{\mathbb{R}}
\newcommand{\bC}{\mathbb{C}}
\newcommand{\bG}{\mathbb{G}}
\newcommand{\bP}{\mathbb{P}}
\newcommand{\bE}{\mathbb{E}}
\newcommand{\bH}{\mathbb{H}}
\newcommand{\bF}{\mathbb{F}}
\newcommand{\bN}{\mathbb{N}}
\newcommand{\naturals}{\mathbb{N}}
\newcommand{\Q}{\mathbb{Q}}
\newcommand{\rationals}{\mathbb{Q}}
\newcommand{\integ}{\mathbb{Z}}

%MATH BOLD:
\newcommand{\bfa}{\mathbf{a}}
\newcommand{\bfb}{\mathbf{b}}
\newcommand{\bfg}{\mathbf{g}}
\newcommand{\bft}{\mathbf{t}}
\newcommand{\bfx}{\mathbf{x}}
\newcommand{\bfy}{\mathbf{y}}
\newcommand{\bfzero}{\mathbf{0}}

%GROUP THEORY:
\newcommand{\inv}{^{-1}}

%FUZZY TOPOLOGY:
\newcommand{\nphi}{\bigodot}

%ALGEBRAIC TOPOLOGY:
\newcommand{\rel}{\text{ rel}}

%HOMTOPY GROUPS:
\newcommand{\htopo}{\mathcal{H}\mathfrak{o}\mathcal{T}\mathfrak{o}\mathfrak{p}_*}
\newcommand{\topo}{\mathcal{T}\mathfrak{o}\mathfrak{p}_*}
\newcommand{\htop}{\mathcal{H}\mathfrak{o}\mathcal{T}\mathfrak{o}\mathfrak{p}}
\newcommand{\cattop}{\mathcal{T}\mathfrak{o}\mathfrak{p}}

%MAPS:
\newcommand{\tends}{\rightarrow}
\newcommand{\id}{\text{id}}

%ANALYSIS:
\newcommand{\dsum}{\displaystyle\sum}
\newcommand{\dprod}{\displaystyle\prod}

%DASH:
\newcommand{\dash}{^\prime}
\newcommand{\ddash}{^{\prime\prime}}
\newcommand{\tdash}{^{\prime\prime\prime}}

%CATEGORY THEORY
\newcommand{\Ob}{\text{Ob}}

%LOGIC
\newcommand{\imp}{\Rightarrow}
\newcommand{\bimp}{\Leftrightarrow}
\newcommand{\Eq}{\text{Eq}}




%HEADINGS:
\newtheorem{thm}{Theorem}
\newtheorem{prop}[thm]{Proposition}
\newtheorem{cor}[thm]{Corollary}
\newtheorem{defn}[thm]{Definition}
\newtheorem{obs}[thm]{Observation}
%\newtheorem{prop}[thm]{Proposition}
\newtheorem{lem}[thm]{Lemma}
\newtheorem{conj}[thm]{Conjecture}
\newtheorem{claim}[thm]{Claim}
\newtheorem{note}[thm]{Note}
\newtheorem{notation}[thm]{Notation}
\newtheorem{rem}[thm]{Remark}
\newtheorem{prob}[thm]{Problem}

%\setmainfont{Cormorant Garamond SemiBold}
\urlstyle{same}
\usepackage{tikzpagenodes}
\usepackage{tikz}
\usepackage{eso-pic}
\begin{document}
\title{Problem set 3}
\author{Anamitro Biswas\\Email: anamitroappu@gmail.com/anamitrob@iitbhilai.ac.in}
\maketitle
\begin{note}
Writing solutions like this in the exams might not fetch you full marks. Please clearly mention the name of any theorem or result you use; and also try to be very specific with $\e$-$\dt$ rigourously in each problem. Don't write one step without proper literal justification from the definition, unless prompted otherwise.
\end{note}
\begin{itemize}
  \item[1.]Show that $\dsum_{n=1}^\infty \dfrac{1-n}{1+2n}$ and $\dsum_{n=1}^\infty \lb -1\rb^n$ diverge. Do $\dsum_{n=1}^\infty \dfrac{n+1}{n+2}$ and $\dsum_{n=1}^\infty\log\lb 1+ \dfrac{1}{n}\rb$ converge?
  \item[\textit{Soln}.]$\la\dfrac{1-n}{1+2n}\ra=\la\dfrac{\dfrac{1}{n}-1}{\dfrac{1}{n}+2}\ra\tends\la-\dfrac{1}{2}\ra\neq 0$, so $\dsum_{n=1}^\infty \dfrac{1-n}{1+2n}$ does not converge.
  
  For even $n\in\naturals$, $\dsum_{n=1}^\infty \lb -1\rb^n=0$, and for odd $n$, $\dsum_{n=1}^\infty \lb -1\rb^n=1$. The sequence of partial sums $\lc\dsum_{i=1}^n \lb -1\rb^i\rc_{n\in\naturals}$ oscillates hence does not converge.
  
  The third one is similar to the first one.
  
 $\dsum_{n=1}^\infty\log\lb 1+ \dfrac{1}{n}\rb=\log\lb\dprod_{n=1}^\infty \lb 1+ \dfrac{1}{n}\rb\rb=\log\lb\dprod_{n=1}^\infty \lb\dfrac{n+1}{n}\rb\rb$. Consider the sequence of partial sums: $\lc\log\lb\dprod_{i=1}^n \lb\dfrac{i+1}{i}\rb\rb\rc_{n\in\naturals}=\lc\log\lb\dfrac{n+1}{1}\rb\rc_{n\in\naturals}=\lc\log\lb n+1\rb\rc_{n\in\naturals}$ which is increasing. Thus the series is divergent. A series like this is called a \emph{telescoping series}, because it just brings the initial and the distant end term close together, skipping the intermediate distance like a telescope does.
  \item[]
    \item[2(i)]$\dsum_{n=1}^\infty\dfrac{(-1)^n}{(2n)!}$.
  \item[\textit{Soln}.]Let $a_n=\dfrac{(-1)^n}{(2n)!}$. Then $\la\dfrac{a_{n+1}}{a_n}\ra=\la\dfrac{1}{(2n+1)(2n+2)}\ra\tends 0$ as $n\tends\infty$. Therefore, the series is convergent.
  \item[]
    \item[2(ii)]$\dsum_{n=1}^\infty\dfrac{(-1)^n}{2^n}$.
  \item[\textit{Soln}.]The series, being absolutely convergent, is also convergent.
  \item[]
    \item[2(iii)]$\dsum_{n=1}^\infty(-1)^{n+1}\lb\dfrac{1}{\sqrt{n}}+\dfrac{(-1)^{n-1}}{n}\rb$.
  \item[\textit{Soln}.]$\dsum_{n=1}^\infty(-1)^{n+1}\lb\dfrac{1}{\sqrt{n}}+\dfrac{(-1)^{n-1}}{n}\rb=\dsum_{n=1}^\infty\lb(-1)^{n+1}\dfrac{1}{\sqrt{n}}+\dfrac{1}{n}\rb$. For $n$ odd, the subseries dominates the series $\dsum\dfrac{1}{n}$ which is divergent. For $n$ odd, all the individual entries are positive. So, the series in all is divergent.
  \item[]
    \item[2(iv)]$\dsum_{n=1}^\infty(-1)^{n+1}\lb\dfrac{n}{2n-1}\rb$.
  \item[\textit{Soln}.]$\dfrac{n}{2n-1}-\dfrac{n+1}{2n+1}=\dfrac{1}{\lb 2n+1\rb\lb 2n-1\rb}$. So, $\dsum_{n=1}^\infty(-1)^{n+1}\lb\dfrac{n}{2n-1}\rb=\dsum_{k=1}^\infty\dfrac{1}{(4k-1)(4k-3)}$, taking $n+1=2k$ and seeing two terms of the series at a time, then the next two terms etc. This shows that the series is divergent.
  
\end{itemize}
\end{document}
\begin{eqnarray*}
\la \displaystyle\sum_{k=1}^n\dfrac{\sin k!}{k(k+1)}-\displaystyle\sum_{k=1}^m\dfrac{\sin k!}{k(k+1)}\ra&=&\la \displaystyle\sum_{k=m+1}^n\dfrac{\sin k!}{k(k+1)}\ra\\
&\leq& \displaystyle\sum_{k=m+1}^n\la\dfrac{\sin k!}{k(k+1)}\ra\\
&\leq& \displaystyle\sum_{k=m+1}^n\la\dfrac{1}{k(k+1)}\ra\\
&\leq& \displaystyle\sum_{k=m+1}^n\la\dfrac{1}{(m+1)(m+2)}\ra\\
&=&\dfrac{n-m}{(m+1)(m+2)}\\
&<&\e.
\end{eqnarray*} $(n>m)$We get, if 
